%%%%%%%%%%%%%%%%%%%%%% pref_it.tex %%%%%%%%%%%%%%%%%%%%%%%%%%%%%%%%%%%%%
%
% Esempio di prefazione
%
% Usare questo file come template per il vostro documento.
%
%%%%%%%%%%%%%%%%%%%%%%%% Springer-Verlag %%%%%%%%%%%%%%%%%%%%%%%%%%

% !TEX encoding = UTF-8 Unicode
% !TEX TS-program = pdflatex
% !TEX spellcheck = it-IT
% !TEX root = ../Ricettine.tex

\preface

Questo ``ricettario'' nasce durante il lockdown globale dovuto al virus COVID 19, questo solo per farvi capire che si tratta di un testo ad alto contenuto di stronzate, generato da un mix fatale di noia, voglia di cazzeggio e delirio (q.b.). Gli autori sono 3 aspiranti ingegneri che in realtà stanno usando tutta questa cosa solo per imparare ad usare Github (il quale ahimè non ha nulla a che fare con il fratello Porn).
Adesso che vi abbiamo messo all'erta sulla vera natura del contenuto possiamo dirvi quale era invece l'idea generata nel delirio di partenza: lo scopo in effetti era scrivere un ricettario un po' romanzato e con qualche trick per fare bene alcune semplici ricette di cucina. Nessuno di noi ha davvero esperienza culinaria, quindi per lo più quello che scriveremo di seguito non è altro che il frutto di un metodo trial and error di qualche anno. Quindi di fatto la dimostrazione che le ricette funzionino realmente è lasciata al lettore cit. . 


%% Si prega di  "firmare" la prefazione
\vspace{1cm}
\begin{flushright}\noindent
Valencia,\hfill {\it Federico Carlini}\\
04, 2020\hfill\\
\end{flushright}


