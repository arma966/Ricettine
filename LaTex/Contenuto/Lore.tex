% !TEX encoding = UTF-8 Unicode
% !TEX TS-program = pdflatex
% !TEX spellcheck = it-IT
% !TEX root = ../Ricettine.tex

\chapter{Colazione}

\chapter{Pranzo}
\section{Gnocchetti alla romana}
\chapter{Cena}

\chapter{Spuntino}
\chapter{Dolci}
\section{Pastiera Napoletana}
\subsection{Ingredienti}
\subsubsection{Per la pasta frolla:}
\begin{itemize}
\item 150gr burro
\item 100gr zucchero
\item 275gr farina
\item 3 tuorli d'uovo
\end{itemize}
\subsubsection{Per la crema:}
\begin{itemize}
\item 3 uova
\item 400gr ricotta
\item 150gr orzo
\item 500ml latte
\item buccia di limone
\item 135gr zucchero
\item aroma fiori d'arancio
\item canditi
\end{itemize}
\subsection{Procedimento}
\subsubsection{Per la pasta frolla}
Con le mani lavorare il burro con lo zucchero, e infine aggiungere la farina, fino ad ottenere un impasto granuloso ma omogeneo. Aggiungere i tuorli, lavorare fino a creare una palla da riporre in frigo avvolta nella pellicola.
\subsubsection{Per la crema}
Cuocere l'orzo nel latte, assieme a 50gr di zucchero e la buccia del limone fino a che non sarà cotto e il latte assorbito. Se risulta non cotto completamente aggiungere un po' d'acqua e continuare la cottura fino a raggiungere la consistenza desiderata. Una volta cotto lasciarlo raffreddare. Una volta raffreddato in una terrina mescolare la ricotta con lo zucchero rimanente (85gr), le uova e il grano cotto. Aggiungere l'aroma fiori d'arancioe i canditi.
\subsubsection{Assemblaggio e cottura}
Stendere la pasta frolla su di un foglio di carta forno e posizionarla sulla teglia. Ritagliare le parti in eccesso, stenderle nuovamente e creare delle striscioline con cui decorare la parte superiore della torta, creando una maglia intrecciata.\\
Cuocere in forno statico per 80 minuti a 180 gradi. Nell'ultima fase di cottura, se la superficie risulta gia scura, coprirla con un foglio di carta d'alluminio e continuare la cottura abbassando leggermente la temperatura.
\pagebreak