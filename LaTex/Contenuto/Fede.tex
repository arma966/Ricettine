
% !TEX encoding = UTF-8 Unicode
% !TEX TS-program = pdflatex
% !TEX spellcheck = it-IT
% !TEX root = ../Ricettine.tex

\chapter{Colazione}

\begin{figure}
\centering
  \includegraphics[width=.3\textwidth]{fig/carlino.png}
\caption{}
\label{}
\end{figure}

\section{Il Caffè}

\chapter{Pranzo}

\section{Carbonara}
%foto

Vabbè direte voi, chi non sa fare la carbonara?\\
Mettiamola sul piano democratico, di questi tempi va molto di moda, e diciamo che tutti noi abbiamo il nostro modo di fare la carbonara. (scrivere da buonista del cazzo è davvero difficile)\\
Tuttavia bando alle ciance, il primo passo sono gli ingredienti. La carbonara ha 4 ingredienti in croce, quindi se volete farla con i n.3 della barilla il grana padano e la pancetta chiudete il pdf e arrangiatevi. La prima cosa imprescendibile è la pasta, potete scegliere pasta lunga o corta basta che sia buona, la seconda cosa è che le correnti di pensiero su alcune cose non possono esistere, la carbonara va fatta con il guanciale e con il pecorino romano.

\subsection{Ingredienti}


\begin{itemize}
\item { \bf Pecorino Romano} (quello con la crosta nera, NON GIA' GRATUGGIATO);
\item {\bf Guanciale} circa 50/70 gr per persona;
\item {\bf Uova} 1 persona $\to$ 2 tuorli, 2 p. $\to$ 2 tuorli e un uovo intero, 3 p. $\to$ 3 tuorli e un uovo intero e così via;
\item {\bf Pasta } dipende da quanto mangiate di solito io direi 120gr a testa; 
\item {\bf Sale };
\item {\bf Pepe }; 
\item {\bf Olio };
\end{itemize}

P.S. Andrebbero solo tuorli, infatti con l'uovo intero si rischia un po' di più l'effetto frittata, io la faccio in questo modo perchè gli albumi li butto e mi dispiace buttarli tutti.

P.P.S. Credo che l'olio si possa anche non mettere, non so se davvero sia utile.

\subsection{Procedimento}
Non è un piatto difficile da fare e nemmeno lungo, in verità tutto il procedimento può essere fatto nel tempo in cui 

\begin{enumerate}
\item Per prima cosa mettete a bollire l'acqua con il sale grosso; in contemporanea mettete il guanciale (tagliato a cubetti) nella padella in cui poi salterete la pasta. La cottura del guanciale è una parte molto importante, deve diventare croccante, ma non deve essere bruciato. %foto

Nota: per una padella da 28cm di diametro consiglio al massimo 3 porzioni di pasta, se siete più persone fai due padelle.

\item Nel tempo in cui si arrosola il guanciale gratuggiate il pecorino e rompete le uova in una scodella. Aggiungete il pecorino alle uova poco per volta e con la forchetta mescolate (non sbattete) in modo che le uova e pecorino formino una crema. Con le ribbe della forchetta schiacciate eventuali grumi o i pezzetti di pecorino in modo che diventino più piccoli. La giusta quantità di pecorino è quella che permette di ottenere una consistenza simile a quella dello yogurt (normale non quello greco).%foto
Aggiungete poco sale, un cucchiaino d'olio d'oliva e pepe abbastanza da fare in modo che l'effetto sia lo stesso del ciccolato nella stracciatella. %foto

\item Buttate la pasta e state molto attenti alla cottura, se avete un mestolo bucato (o delle pinze) per scolarla un poco alla volta è meglio, altrimenti poco prima di scolarla salvate un bicchiere di acqua di cottura. La pasta scolata va direttamente sulla padella con il guanciale (fiamma spenta ma padella calda). 
A questo punto la parte più difficile, aggiungete la crema di uovo e pecorino e subito dopo un po' di acqua di cottura ed iniziate subito a saltare la pasta, è fondamentale non perdere nemmeno un secondo.
Inizialmente l'impressione sarà quella di aver fatto un casino perchè la carbonara sembrerà liquida. Continuate a saltare finchè tutto non si sarà amalgamato.
Risottare la pasta con l'acqua di cottura funziona al 100\%, anche se avete proprio fatto un casino e avete messo troppa acqua. In questo caso accendete la fiamma e non smettete mai di saltare finchè non sarà cremosa al punto giusto. 
%da finire

\end{enumerate}

\chapter{Cena}

\section{Vellutate}

\chapter{Spuntino}

\pagebreak
